\documentclass{article}
\usepackage[utf8]{inputenc}
\usepackage{amsmath,color,amssymb,amsthm,mathrsfs,verbatim,tikz,graphicx}
\usepackage[margin=2.5cm]{geometry}
\usepackage{xcolor}
\usepackage{enumerate}
\usetikzlibrary{matrix,arrows,decorations.pathmorphing}
\theoremstyle{definition}
\newtheorem{defn}{Definition}
\newtheorem*{fact}{Fact}
\newtheorem{example}{Example}
\newtheorem*{ex}{Exercise}
\newtheorem*{soln}{Solution}
\newtheorem*{prob}{Problem}
\newtheorem*{lemma}{Lemma}

\theoremstyle{theorem}
\newtheorem{thm}{Theorem}

\newcommand{\R}{\mathbb{R}}
\newcommand{\A}{\mathbb{A}}
\newcommand{\Q}{\mathbb{Q}}
\newcommand{\Z}{\mathbb{Z}}
\newcommand{\X}{\mathbb{X}}
\newcommand{\Y}{\mathbb{Y}}
\newcommand{\J}{\mathbb{J}}
\newcommand{\N}{\mathbb{N}}
\newcommand{\M}{\mathbb{M}}
\newcommand{\C}{\mathbb{C}}
\newcommand{\K}{\mathbb{K}}
\renewcommand{\S}{\mathbb{S}}
\newcommand{\E}{\mathbb{\emptyset}}
\newcommand{\F}{\mathbb{F}}
\newcommand{\Proj}{\mathbb{P}}
\newcommand{\HP}{\mathbb{H}}
\newcommand{\D}{\mathbb{D}}
\newcommand{\Pic}{\mbox{Pic}}
\newcommand{\Div}{\mbox{Div}}
\newcommand{\T}{\mathcal{T}}
\newcommand{\atan}{\operatorname{atan2}}
\newcommand{\acos}{\operatorname{acos}}


\begin{document}

\title{Advanced Calculus HW 10 - Due October 27, 4pm}
\author{Luis Berlioz}
\maketitle



\begin{prob}[\#11 page 199]
    Let $f:\ (a,b) \to \R$ be given.
    \begin{enumerate}[(a)]
        \item If $f''(x)$ exists, prove that:
            $$\lim_{h\to 0} \frac{f(x-h) - 2 f(x) + f(x+h)}{h^2} = f''(x)$$
        \item Find an example that this limit can even when $f''(x)$ fails to exist.
    \end{enumerate}
\end{prob}
\begin{soln}
    \begin{enumerate}[(a)]
        \item By the Taylor Approximation Theorem:
            \begin{gather*}
                f(x+h) = f(x) +f'(x)h+f''(x)h^2/2 + R_1(h)\\
                f(x-h) = f(x) -f'(x)h+f''(x)h^2/2 + R_2(h)
            \end{gather*}
            Adding these two equation gives: 
            $$f(x+h) - 2f(x) +f(x-h) -R_1(h) - R_2(h) = f''(x)h^2$$
            Using that for $i=1,2$, $R_i(h)/h^2 \to 0$ when $h\to 0$ we get:
            $$\lim_{h\to 0} \frac{f(x-h) - 2 f(x) + f(x+h)}{h^2} = f''(x)$$
        \item Let $f(x) = |x|x$ for all $x\in \R$, then $f'(x) =2|x|$  and therefor $f''$ does not exist at $x=0$. Nevertheless, the limit does exist:
            $$\lim_{h\to 0} \frac{f(0-h) - 2 f(0) + f(0+h)}{h^2} = \lim_{h\to 0 } \frac{-h^2+h^2}{h^2} = 0$$
    \end{enumerate}
\end{soln}
\vspace{1in}



\begin{prob}[\#16 page 199]
    $\log(x)$ is defined to be $\int_1^x 1/t\, dt$ for $x>0$. Using only the mathematics explained in this chapter,
    \begin{enumerate}[(a)]
        \item Prove that $\log$ is a smooth function.
        \item Prove that $\log(x\,y) = \log x + \log y$ for all $x,y>0$. [Hint: Fix $y$ and define $f(x) = \log(x\, y) - \log x - \log y$. Show that $f(x) \equiv 0$.]
        \item Prove that $\log$ is strictly monotone increasing and its range is all of $\R$.
    \end{enumerate}
\end{prob}
\begin{soln}
    \begin{enumerate}[(a)]
        \item By the Fundamental Theorem of Calculus and since $1/t$ is continuous for all $t> 0$:
            $$\frac d{dx} \log x = 1/x$$

        \item First we show that $d\, x^n/dx = n x^{n-1 }$ by induction. For $n=1$, it is a linear function with constant slope 1. Assume that $d\, x^n/dx = n x^{n-1 }$, then using the Leibniz Formula:
            $$\frac{d\, x^{n+1}}{dx^n} = \frac{d\, x\, x^{n}}{dx^n} = x^n + x \, n\,x^{n-1 } = (n+1)x^n$$
            Next,  by the quotient rule:
            $$\left(\frac 1{x^n}\right)' = \frac{-n\,x^{n-1 }}{x^{2n }}= \frac{-n}{x^{n+1 }}$$
            Then for $n\geq 1$:
            $$\frac{d^{n }\, f(x)}{dx} = \frac{(-1)^{n+1 }(n-1)!}{x^n}$$
            Therefore, $f$ is a smooth function on $x>0$.
        \item Following the hint:
            $$\frac{d\, \log(x\,y)}{dx} = \frac y{x\, y} - \frac 1x = 0$$
            Then $\log (y\,x)- \log(x) -\log(y)$ is constant. And since it is zero when $x=1$; it is zero for all $x> 0$.

        For all $x> 1$, $1/x>0$, therefore $\log(x)$ is monotone increasing. 

            This means that $\log(2)>\log(1)=0$. And this implies that $\log(2^n) = n\log(2)$ for all $n\in\Z$ is unbounded. Then for all $y\in  \R$ there exists $n\in \Z$ such that $n\log(2) \leq y \leq (n+1)\log(2)$. Finally $\log(x)$ is an antiderivative and so it is continuous. Then it satisfies the intermediate value property and there exists $x\in \R$ such that $\log(x)=y$.

    \end{enumerate}
 
\end{soln}
\vspace{1in}


\begin{prob}[\# 40  page 205]
    Set:
    $$ f(x) = \begin{cases} 0 & \text{ if } x\leq 0\\
    \sin\frac \pi x & \text{ if } x>0\end{cases}\quad \text{ and }\quad g(x) =\begin{cases} 0 & \text{ if } x\leq 0\\
    1 & \text{ if } x>0\end{cases}  $$ 
    Prove that $f$ has an antiderivative but $g$ does not.
\end{prob}
\begin{soln}
    Observe that $f$ is bounded and it is only discontinuous at $x=0$. This means that it satisfies the hypothesis of the Lebesgue-Riemann theorem. Then $f$ is an integrable function and thus by the fundamental theorem of calculus it has an antiderivative. 

    On the other hand if $g$ were to have an antiderivative, then it would satisfy the intermediate value property. Since it does not for $0<y<1$ then it does not have an antiderivative.
\end{soln}
\vspace{1in}


\begin{prob}[\# 36  page 204]
    Generalizing Exercise 1.31, we say that $f:\ (a,b) \to \R$ has a \textbf{jump discontinuity} (or a discontiuity of the \textbf{first kind}) at $c\in (a,b)$ if:
    $$f(c^-) = \lim_{x\to c^- }f(x) \quad \text{and} f(c^+) = \lim_{x\to c^+ }f(x)$$
    exist, but are either unequal or are unequal to $f(c)$. (The three quantities exist and are equal if and only if $f$ is continuous at c.) An \textbf{oscillating discontinuity } (or a discontinuity of the \textbf{second kind} is any nonjump discontinuity.
    \begin{enumerate}[(a)]
        \item Show that $f:\ \R \to \R$ has at most countable many jump discontinuities.
        \item What about the function
            $$f(x) = \begin{cases} \sin \frac 1x & \text{if } x>0 \\
            0 & \text{if } x\leq 0 \end{cases}$$
        \item What about the characteristic function of the rationals?
    \end{enumerate}
\end{prob}
\begin{soln}
    \begin{enumerate}[(a)]
        \item If $f$ has a jump discontinuity at $c$, then $f(c^-) < f(c^+)$ or $f(c^+) < f(c^-)$. In either case, there exists a rational number $q$ between these two values. Next, take $N\in \N$ such that if $n>\N$ then:
            $$f(c-1/n) < q < f(c+1/n) \quad \text{or} \quad f(c-1/n) > q > f(c+1/n)$$
            The pairs $(N,q)$ enables to individualize each discontinuity. Since the set $(N,q)\in \N\times \Q$ is countable, then the set of all discontinuities is also countable.
        \item The function $f$ has an oscillating discontinuity at zero and is contuous on the rest of the real line. This is because $f(0^+)$ does not exist since for all $\delta>0$ there are $0<x_1<x_2<\delta$ such that $f(x_1)=1$ and $f(x_2)=-1$.
        \item Since both the set of rationals and irrationals are dense in $\R$, the limits at every point fail to exist. This means that there is an oscillating discontinuity at every point. 
    \end{enumerate}
\end{soln}
\vspace{1in}

\begin{prob}[\# 62  page 208]
    Prove that if the terms of a sequence decrease monotonically, $a_1 \geq a_2\geq \ldots$, and converge to 0 then the series $\sum a_k$ converges if and only if the associated dyadic series:
    $$a_1 + 2a_2 +4a_4 +8a_8 + \cdots = \sum 2^k a_{2^k }$$
    converges. (I call this the \textbf{block test} because it groups the terms of the series in blocks of length $2^{k-1 }$.)
\end{prob}
\begin{soln}
    That the $a_k$ are a monotinically decreasing sequence implies that for all $k>0$:
    $$ \sum_{j= 2^{k-1 }}^{2^k-1 }a_j \geq \sum_{j=2^{k-1 } }^{2^k-1 }a_{2^k } = (2^{k-1 }-1)a_{2^k }$$
    We call $s_k=\sum_{j= 2^{k-1 }}^{2^k-1 }a_j$ (this is the sum on the left above). Then for all $k>0$:
    $$2(s_k + a_{2^k }) \geq 2^ka_{2^k }$$
    Observe that $\sum s_k = \sum a_k$ (where we sum over $k>0$). Then if we assume that $\sum a_k$ converges then $\sum a_{2^k }$ also converges since it is a subset of an absolute converging sequence. Therefore  $\sum 2^k a_{2^k }$ converges.

    On the other hand, for $k >0$:
    $$a_{2^k }2^k \geq \sum_{j=2^k }^{2^{k+1 } }a_j \geq \sum_{j=2^k }^{2^{k+1 }-1 }a_j = r_k $$
    Assuming that $\sum a_{2^k }2^k $ converges implies that $\sum r_k = \sum a_k$ converges.
\end{soln}
\vspace{1in}

\end{document}








