\documentclass{article}
\usepackage[utf8]{inputenc}
\usepackage{enumerate}
\usepackage{amsmath,color,amssymb,amsthm,mathrsfs,verbatim,tikz,graphicx}
\usepackage[margin=2.5cm]{geometry}
\usetikzlibrary{matrix,arrows,decorations.pathmorphing}
\theoremstyle{definition}
\newtheorem{defn}{Definition}
\newtheorem*{fact}{Fact}
\newtheorem{example}{Example}
\newtheorem*{ex}{Exercise}
\newtheorem*{soln}{Solution}
\newtheorem*{prob}{Problem}
\newtheorem*{lemma}{Lemma}

\theoremstyle{theorem}
\newtheorem{thm}{Theorem}

\newcommand{\R}{\mathbb{R}}
\newcommand{\A}{\mathbb{A}}
\newcommand{\Q}{\mathbb{Q}}
\newcommand{\Z}{\mathbb{Z}}
\newcommand{\N}{\mathbb{N}}
\newcommand{\C}{\mathbb{C}}
\newcommand{\E}{\emptyset}
\newcommand{\F}{\mathbb{F}}
\newcommand{\Proj}{\mathbb{P}}
\newcommand{\HP}{\mathbb{H}}
\newcommand{\D}{\mathbb{D}}
\newcommand{\Pic}{\mbox{Pic}}
\newcommand{\Div}{\mbox{Div}}
\newcommand{\T}{\mathcal{T}}

\begin{document}

\title{Advanced Calculus HW 5 - Due October 9, 4pm}
\author{Luis Berlioz}
\maketitle




\begin{prob}[1: Pugh \#18 p. 126]
Is $\R$ homeomorphic to $\Q$? Explain.
\end{prob}
\begin{soln}
    No. We know that $\R$ is connected (we showed it in class). On the other hand, $\Q$ is not, as an example consider the sets $\Q\cap (-\infty, \sqrt 2)$ and $\Q\cap(\sqrt 2, \infty)$, these two set are open in the relative topology of $\Q$, hence, it is not connected. A homeomorphism would preserve the property of either being connected or not. Therefore they cannot be homeomorphic.
\end{soln}
\vspace{1in}


\begin{prob}[2: Pugh \#20 p. 126]
    What function (given by a formula) is a homeomorphism from (-1,1) to $\R$? is every open interval homeomorphic to (0,1)? Why or why not?
\end{prob}
\begin{soln}
    For a homeomorphism between $(-1, 1)$ and $\R$, consider:  $f: (-1,1) \to \R$ given by $f(x) = \tan(\frac\pi2 x)$. This function is continuous on all its domain because it is differentiable. Also, $\lim_{x\to -1 }f(x) = -\infty$ and $\lim_{x\to 1 }f(x) = \infty$. Lastly, its inverse, namely $f^{-1 }(x) = \frac 2\pi \tan^{-1 }(x)$ is also differentiable and thus continuous. 

    For any open interval $(a,b)$ the linear function:
    $$f(x) = \frac{x-a}{b-a}$$  
    Is a homeomorphism between the open intervals $(a,b)$ and $(0,1)$ because it is continuous and with continuous inverse:
    $$f^{-1 }(x) = (b-a)x + a$$
\end{soln}
\vspace{1in}


\begin{prob}[3: Pugh \#22 p. 126]
    If every closed and bounded subset of a metric space $M$ is compact, does it follow that $M$ is complete? (Proof or counterexample)
\end{prob}
\begin{soln}
    This is true, to prove it consider a Cauchy sequence $\{x_n  \}$. As a subset of $M$ the closure of $\overline{\{x_n  \}}$ is closed and bounded (Cauchy sequences are always bounded). In a metric space, compact and sequentially compact are equivalent, so $\{x_n  \}$ has a converging subsequence converging to say, $x\in M$. And since it is Cauchy, this limit is unique. Hence, $M$ is complete.
\end{soln}
\vspace{1in}



\begin{prob}[4: Pugh \#23 p. 126]
    $(0,1)$ is an open subset of $\R$ but not of $\R^2$, when we think of $\R$ as the $x-$axis in $\R^2$. Prove this.
\end{prob}
\begin{soln}
    Let $B(x,\epsilon)\subset \R^{2 }$ be an open ball in the plane with center $0<x<1$ and radius $\epsilon >0$. Consider the point $p = (x,\epsilon/2)$. Note that $p\in B(x,\epsilon)$ but $p$ is not in $(0,1)$. If this set were open then there would exist an $\epsilon >0$ such that the ball $B(x,\epsilon)$ would be contained inside it; since this is not the case, we conclude that it is not open.
\end{soln}
\vspace{1in}


\begin{prob}[5: Pugh \#24 p. 126]
    For which intervals $[a,b]$ in $\R$ is the intersection $[a,b]\cap \Q$ a clopen subset of the metric space $\Q$?
\end{prob}
\begin{soln}
    The set $[a,b]\cap \Q$ is clopen whenever $a,b$ are irrationals. This is because if $a,b$ are irrational numbers, the sets:
    $$(a,b)\cap \Q \text{ and } [a,b] \cap \Q$$
    Are indistinguishable, and therefore clopen in the relative topology of $\Q$.
\end{soln}
\vspace{1in}
\begin{prob}[6: Pugh \#24 p. 126]
    If $S,T \subset M$, a metric space, and $S\subset T$, prove that 
    \begin{enumerate}[(a)]
        \item $\bar S \subset \bar T$.
        \item $int(S) \subset int(T)$.
    \end{enumerate}
\end{prob}
\begin{soln}
    (a) If $x\in \bar S$, then all neighborhood $x\in O$ are NOT disjoint with $S$. Since $S\subset T$ we also get that $O\cap T \neq \E$. Hence, $x\in \bar T$.

    (b) Analogously,  if $x\in int(S)$, then there exists a neighborhood $x\in O\subset S$. Since $S\subset T$ we also get that $O\subset T $. Hence, $x\in int( T)$.
\end{soln}

\end{document}
