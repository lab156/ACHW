\documentclass{article}
\usepackage[utf8]{inputenc}
\usepackage{amsmath,color,amssymb,amsthm,mathrsfs,verbatim,tikz,graphicx}
\usepackage[margin=2.5cm]{geometry}
\usepackage{xcolor}
\usepackage{enumerate}
\usetikzlibrary{matrix,arrows,decorations.pathmorphing}
\theoremstyle{definition}
\newtheorem{defn}{Definition}
\newtheorem*{fact}{Fact}
\newtheorem{example}{Example}
\newtheorem*{ex}{Exercise}
\newtheorem*{soln}{Solution}
\newtheorem*{prob}{Problem}
\newtheorem*{lemma}{Lemma}

\theoremstyle{theorem}
\newtheorem{thm}{Theorem}

\newcommand{\R}{\mathbb{R}}
\newcommand{\A}{\mathbb{A}}
\newcommand{\Q}{\mathbb{Q}}
\newcommand{\Z}{\mathbb{Z}}
\newcommand{\X}{\mathbb{X}}
\newcommand{\Y}{\mathbb{Y}}
\newcommand{\J}{\mathbb{J}}
\newcommand{\N}{\mathbb{N}}
\newcommand{\M}{\mathbb{M}}
\newcommand{\C}{\mathbb{C}}
\newcommand{\K}{\mathbb{K}}
\renewcommand{\S}{\mathbb{S}}
\newcommand{\E}{\mathbb{\emptyset}}
\newcommand{\F}{\mathbb{F}}
\newcommand{\Proj}{\mathbb{P}}
\newcommand{\HP}{\mathbb{H}}
\newcommand{\D}{\mathbb{D}}
\newcommand{\Pic}{\mbox{Pic}}
\newcommand{\Div}{\mbox{Div}}
\newcommand{\T}{\mathcal{T}}
\newcommand{\atan}{\operatorname{atan2}}
\newcommand{\acos}{\operatorname{acos}}


\begin{document}

\title{Advanced Calculus HW 11 - Due December 1, 4pm}
\author{Luis Berlioz}
\maketitle


\begin{prob}[\#1 ]
     Using appropriate Darboux sums calculate (with proof) the  integrals of
$\sin(x)$ and $\cos(x)$ taken over an interval $[0, a]$, where $0 < 2a < \pi$.  (Hint: +Euler)
\end{prob}
\begin{soln}
    To compute these sums, consider the partition with endpoints $\{0,\frac \pi n, \frac {2\pi}n, \ldots , \pi\}$ next we consider the sum:
    $$\sum_{k=0 }^{n-1 }e^{i\frac {k\pi}n }\frac \pi n = \frac {e^{i n\pi/n }-1}{n/\pi(e^{i\frac \pi n -1 })} = \frac {-2}{\frac n\pi(e^{i\frac \pi n -1 })} $$
    Using fundamental limits from calculus 1, we obtain that when $n\to \infty$:
    \begin{gather}
        \frac n\pi (\cos\frac \pi n - 1) \to 0\\
        \frac n\pi (i\sin(\pi n)) \to i
    \end{gather}
    This means that the Darboux sum:
$$\sum_{k=0 }^{n-1 }e^{i\frac {k\pi}n }\frac \pi n \to -2/i =2i$$
Finally comparing real and imaginary parts we get that:
    \begin{gather}
        \int_0^\pi \cos(x)dx = 0\\
        \int_0^\pi \sin(x) dx = 2
    \end{gather}

\end{soln}


\begin{prob}[\#9 page 198]
    Assume that $f: \R \to \R$ is differentiable.
    \begin{enumerate}[(a)]
        \item If there is an $L<1$ such that for each $x\in \R$ we have $f'(x)<L$, prove that there exists a unique point $x$ such that $f(x) =x$. [$x$ is a fixed point for $f$.]        
        \item Show by example that (a) fails if $L=1$.
    \end{enumerate}
\end{prob}
\begin{soln}
    \begin{enumerate}[(a)]
        \item If $f(0)=0$, then $x=0$ is a fixed point of $f$. In the case $f(0)>0$, then by the mean value theorem for all $x>0$ there exists a $0< c< x$ such that:
            $$f(x) - f(0) = f'(c)x < Lx < x$$
            This implies that:
            $$f(x) -x < f(0) +(L-1)x$$
            This shows that  in the interval $[0,f(0)/(1-L)]$, $f(x) -x$ is going from the positive value of $f(0)$ to some negative value at $x=f(0)/(1-L)$. By the intermediate value theorem, which $f(x) - x$ satisfies since it is continuous, $f$ must have a fixed point.

            Similarly for the case $f(0)<0$. For any $x<0$, by the mean value theorem:
            $$f(x) - f(0) = f'(c)x > Lx > x$$
            This implies:
            $$x-f(x) < (L-1)x -f(0)$$
            Therefore, $f$ has a fixed point in the interval $[f(0)/(1-L), 0]$.

            In every case the fixed point of $f$ is  unique because $(f(x)-x)' < L-1<0$. This means that the function is one to one so $f(x)-x$ can only be equal to zero once.

        \item Consider $f(x) = e^{-x } +x$. Then $f'(x) = -e^{-x }+1<1$ for all $x\in \R$. Observe that $f$ has no fixed point since the function: $f(x)-x = e^{-x }$ has no roots.
    \end{enumerate}
\end{soln}
\vspace{1in}



\begin{prob}[\#28 page 203   ]
    Suppose that $Z\subset \R$. Prove that the following are equivalent.
    \begin{enumerate}[(i)]
        \item $Z$ is a zero set.
        \item For each $\epsilon>0$ there is a countable covering of $Z$ by intervals $[a_i, b_i]$ with total length $\sum b_i - a_i <\epsilon$.
        \item For each $\epsilon>0$ there is a countable covering of $Z$ by set $S_i$ with total diameter $\sum diam\, S_i <\epsilon$.

    \end{enumerate}
\end{prob}
\begin{soln}
\begin{description}
    \item[(i) $\implies$ (ii)] The length of the open interval $]a,b[$ is the same as the length of the closed interval with the same limits $[a,b]$. This implies that for every $\epsilon >0$ $Z$ can be covered by open intervals $]a_i,b_i[$ for $i\in \N$ such that:
        $$\sum_{i\in \N }(b_i-a_i)<\epsilon $$
        The closed intervals $[a_i,b_i]$ for all $i\in \N$ also cover $Z$ and their lengths is also less than $\epsilon$.
    \item[(ii) $\implies$ (iii)] The diameter of a the closed intervals $S_i = [a_i, b_i]$ is $b_i -a_i$. If we take the same collection of intervals $[a_i,b_i]$ as above then we have a countable covering $S_i$ for all $i\in \N$ such that $\sum diam\, S_i <\epsilon$.
    \item[(iii)$\implies$ (i)] For any covering of $Z$ by sets $S_i$ for all $i\in \N$, for every $z\in Z$ there exists $S_j$ such that $z\in S_j$.  Let $a_j= \inf S_j$ and $b_j = \sup S_j$. This means that $diam \, S_j = b_j-a_j$ and that $S_j \subset ]a_j, b_j[$. Lastly, by hypothesis for all $\epsilon >0$  there exists a covering $S_i$  and sequences $a_i$ and $b_i$ defined as above such that:
        $$\sum (b_i - a_i) = \sum diam\, S_i < \epsilon$$
\end{description}
\end{soln}
\vspace{1in}

\begin{prob}[\# 31  page 203]
    Define the Cantor set by removing from [0,1] the middle interval of length 1/4. From the remaining two intervals $F^1$ remove the middle intervals. From the remaining four intervals $F^2$ remove the middle intervals of length 1/64, and so on. At the $n^{\text{th} }$ step in the contruction of $F^n$ consists of $2^n$ subintervals of $F^{n-1 }$.
    \begin{enumerate}[(a)]
        \item Prove that $F = \cap F^n$ is a Cantor set but not a zero set. It is referred to as a \textbf{fat Cantor set}.
        \item Infer that being a zero set is not a topological property: If two sets are homeomorphic and one is a zero set then the other need not be a zero set. [Hint: To get a sense of this fat Cantor set, calculate the total length of the intervals which comprise its complement.]
    \end{enumerate}
    \end{prob}
            \begin{soln}
    \begin{enumerate}[(a)]
        \item            To show that $F$ is a Cantor set, we need to show it is compact, infinite, perfect and totally disconnected.
                \begin{description}
                    \item[Compact:] Each $F^n$ is compact for all $n\in \N$, the intersection of arbitrary amount of compact set is compact, then $F$ is compact.
                    \item[Infinite:] The endpoints of the intervals that comprise $F^n$ are an infinite set contained in all $F^n$: $E = \{ 0, 1, 3/8, 5/8,\ldots \}$.
                    \item[Perfect:] For any $\epsilon>0$ take $n\in \N$ such that the intervals comprising $F^n$ have length less than $\epsilon$. Then if $x\in F$ is contained in some interval $I$ of $F^n$ the intersection of the set of endpoints $E$ and $I$ is infinite. This means that $F$ ``clusters around $x$''.

                    \item[Totally Disconnected:] Every point in $F$ is contained in all the intervals that form $F^n$, this intervals are relatively clopen as subsets of $[0,1]$, since these intervals have arbitrarily short length, $F$ is totally disconnected.
                \end{description}

                To compute the measure of $F$, we compute the measure of its complement and substract it from 1 which is the measure of $[0,1]$:
                $$1 - \sum_{k=0 }^\infty \frac{2^k}{4^{k+1 }} = 1-1/4(2) = 1/2$$
                Then $F$ does not have zero measure.

            \item According to Theorem 73 Moore-Kline in the textbook, All Cantor sets are homeomorphic. This implies that the Cantor middle-third $C$   and the Fat Cantor set $F$ are homeomorphic. Since $C$ is a zero measure set and $F$ is not, we conclude that a homeomorphism is not preserving the zero measure property. A homeomorphism preserves all topological properties so having zero measure is not topological.
    \end{enumerate}
            \end{soln}
\vspace{1in}

\begin{prob}[\# 37  page 205]
    Suppose that $f: \R \to \R$ has no jump dicontinuities. Does $f$ have the intermediate value property? (Proof of counterexample)
\end{prob}
\begin{soln}
    Consider the counterexample: $f(x) =sgn(x) + 1/2\sin(1/x)$ when $x\neq 0$ and $f(0)=0$. This function is continuous on all $\R$ except for $x=0$. And at  this point it has an oscillating discontinuity because we have shown in class that the limit when $x\to 0^+$ or $0^-$ of functions similar to $\sin(1/x)$ does not exist. Thus, we conclude that $f$ has no jump discontinuities. Also, $f$ does not have the intermediate value property since even though it has 0  and  1 in its range, it does not go through every number in [0,1], for example 1/4.
\end{soln}
\vspace{1in}


\begin{prob}[\# 46  page 206]
    \begin{enumerate}[(a)]
        \item Prove that the integral of the Zeno's staircase function described on page 174 is 2/3.
        \item What about the Devil's staircase?
    \end{enumerate}
\end{prob}
\begin{soln}
    \begin{enumerate}[(a)]
        \item Let $f(x)$ be the Zeno's staircase function, then it has a countable set of point where it is discontinuous, namely $\{1 - 1/2^n :\ n\in \N \}$. This means that the function is Riemann Integrable. To find the integral we sum up the areas under which the function is constant:
            $$\sum_{n=1 }^\infty \left(1-\frac 1{2^n}\right)\frac 1{2^n}= \sum_{n=1 }^\infty\frac 1{2^n} - \frac 1{4^n}$$
            And since both these series are convergent they can be summed individually to get $1- 1/3=2/3$.
        \item Using the terminology in the textbook, the devil's staircase is given by the function:
            $$H(x)=\sum_{i=1 }^\infty \frac{\omega_i/2}{2^i}$$
            When $x$ is in the Cantor set $C$. And $H$ has constant value at the discarded gap intervals. Also, $H$ is continuous therefore it is integrable. 

            To find the value of the integral, observe that for all $0\leq x \leq 1$, $H(x)+ H(1-x) =1$. If $x\in C$ then:
            $$H(x)+ H(1-x) = \sum_{i=1 }^\infty \frac{\omega_i/2}{2^i} + \sum_{i=1 }^\infty \frac{(2-\omega_i)/2}{2^i} = \sum_{i=1 }^\infty \frac{1}{2^i}= 1$$
            If $x$ is not in $C$ then it is in one of the discarded intervals which has endpoints that can be written in base 3 using only the symbols 0 and 2 so the above argument also applies. Finally, the integral is then:
            $$\int_0^1 H(x) dx + \int_0^1 H(1-x)dx = \int_0^1 1dx = 1$$
            Doing the change of variable: $u = 1-x$, $du = -dx$:
            $$\int_0^1 H(x) dx - \int_1^0 H(u)du = 2\int_0^1 H(x)dx = 1$$
And we get that the integral is 1/2.
    \end{enumerate}
\end{soln}
\vspace{1in}

\begin{prob}[\# 51  page 207]
    If $f,g$ are Riemann integrable on $[a,b]$,  and $f(x) < g(x)$ for all $x\in [a,b]$, prove that $\int_a^b f(x)dx < \int_a^bg(x)dx$. (Note the \textit{strict} inequality.)
\end{prob}
\begin{soln}
    Assume, to get a contradiction, that $\int_a^b(g(x)-f(x))dx = 0$. Then by corollary 31, $g(x) = f(x)$ almost everywhere. This is  a contradiction since the empty set has measure zero. 

    This means that $\int_a^b(g(x)-f(x))dx > 0$ equivalently:
    $$\int_a^bg(x)dx > \int_a^bf(x)dx = 0$$
\end{soln}
\vspace{1in}

\begin{prob}[\# 59  page 208]
    Prove that if $a_n\geq 0$ and $\sum a_n$ converges then $\sum(\sqrt{a_n})/n$ converges.
\end{prob}
\begin{soln}
   First note that for all $n\in \N$
    $$0\leq (\sqrt{a_n}-1/\sqrt n)^2 = a_n - 2\sqrt{a_n}/n + 1/n^2$$
    Then:
    $$\sqrt{a_n}/n\leq 1/2(a_n +1/n^2)$$
    Euler showed in 1734, that $\sum 1/n^2$ converges then $\sqrt{a_n}/n$ also converges.
\end{soln}
\vspace{1in}

\begin{prob}[\# 60  page 208]
    \begin{enumerate}[(a)]
        \item If $\sum a_n$ converges and $(b_n)$ is monotonic and bounded, prove that $\sum a_nb_n$ converges.
        \item If the monotonicity condition is dropped, or replaced by the assumption that $\lim_{n\to \infty }b_n =0$, find a counterexample to convergence of $\sum a_n b_n$.
    \end{enumerate}
\end{prob}
\begin{soln}
    \begin{enumerate}[(a)]
        \item Observe that $(b_n)$ converges because it is monotonic and bounded. In particular this means that $(b_n)$ is a Cauchy sequence.
           
            We will call $A_n^m = a_n+ a_{n+1 }+\ldots + a_m$ for all $n,m\in\N$ (with $n\leq m$ ). Next we show that:
            $$\sum_{k=n }^m A_n^k (b_k-b_{k+1 }) + b_{m+1 }A_n^m  = \sum_{k=n }^{m } a_kb_k$$
            We prove this by induction over $j = m-n$. When $j=0$, that is, $m=n$, we get:
            $$a_n(b_n-b_{n+1 })+a_nb_{n+1 } = a_n b_n - a_nb_{n+1 } +a_nb_{n+1 }=a_nb_n$$
            Next we assume the result for  $m-n = j$ and consider:
            $$\sum_{k=n }^{m+1 }A_n^k(b_k-b_{k+1 }) + b_{m+2 }A_n^{m+1 }$$
            $$=\sum_{k=n }^{m }A_n^k(b_k-b_{k+1 }) + A_n^{m+1 }(b_{m+1 }- b_{m+2 })  + b_{m+2 }A_n^{m+1 } $$
            $$=\sum_{k=n }^{m }A_n^k(b_k-b_{k+1 }) + A_n^{m+1 }b_{m+1 } = \sum_{k=n }^ma_kb_k$$
            With this result, we proceed to show that $\sum_{k=n }^ma_kb_k$ is a Cauchy sequence and then the series $\sum_{k=1 }^\infty a_kb_k$ converges.

            Take any $\epsilon >0$ and $N\in \N$ such that if $n,m>N$ then By the Cauchy criterion for sums:
            $$|A_n^m| = \left|\sum_{k=n }^m a_k \right| <\epsilon \quad \text{and} \quad |b_k|<b \text{ for all }k\in \N$$
            Then 
            $$\left| \sum_{k=n }^m a_k b_k\right| = \left| \sum_{k=n }^m A_n^k (b_k-b_{k+1 }) + b_{m+1 }A_n^m \right| \leq \epsilon|b_n-b_{m+1 }| + \epsilon b \leq 3\epsilon$$ 
    \end{enumerate}

\end{soln}
\vspace{1in}

\begin{prob}[\# 64  page 208]
    Concoct a series $\sum a_k$ such that $(-1)^ka_k>0, \ a_k\to 0$, but the series diverges.
\end{prob}
\begin{soln}
Consider the sequence:
    $$a_n = \begin{cases} \frac{1}{(n-1)/2}, n\text{ is odd}\\
    -e^{-n/2}, n\text{ is even} \end{cases} $$
    Then the sign of $a_k$ alternates as $k$ changes from even to odd. Also, $b_k\to 0$ when $k\to \infty$. The series can be written as:
    $$\sum a_k = \sum_{n=1}^\infty \left( \frac{1}{n} - e^{n}\right)$$
    Which diverges because the integral diverges:
    $$\int_1^\infty \left( 1/x -e^{-x}\right)dx = \left. \left( \ln x +e^{-x}\right)\right|_1^\infty= \infty $$
\end{soln}
\vspace{1in}

\end{document}








