\documentclass{article}
\usepackage[utf8]{inputenc}
\usepackage{amsmath,color,amssymb,amsthm,mathrsfs,verbatim,tikz,graphicx}
\usepackage[margin=2.5cm]{geometry}
\usepackage{xcolor}
\usepackage{enumerate}
\usetikzlibrary{matrix,arrows,decorations.pathmorphing}
\theoremstyle{definition}
\newtheorem{defn}{Definition}
\newtheorem*{fact}{Fact}
\newtheorem{example}{Example}
\newtheorem*{ex}{Exercise}
\newtheorem*{soln}{Solution}
\newtheorem*{prob}{Problem}
\newtheorem*{lemma}{Lemma}

\theoremstyle{theorem}
\newtheorem{thm}{Theorem}

\newcommand{\R}{\mathbb{R}}
\newcommand{\A}{\mathbb{A}}
\newcommand{\Q}{\mathbb{Q}}
\newcommand{\Z}{\mathbb{Z}}
\newcommand{\X}{\mathbb{X}}
\newcommand{\Y}{\mathbb{Y}}
\newcommand{\J}{\mathbb{J}}
\newcommand{\N}{\mathbb{N}}
\newcommand{\M}{\mathbb{M}}
\newcommand{\C}{\mathbb{C}}
\newcommand{\K}{\mathbb{K}}
\renewcommand{\S}{\mathbb{S}}
\newcommand{\E}{\mathbb{\emptyset}}
\newcommand{\F}{\mathbb{F}}
\newcommand{\Proj}{\mathbb{P}}
\newcommand{\HP}{\mathbb{H}}
\newcommand{\D}{\mathbb{D}}
\newcommand{\Pic}{\mbox{Pic}}
\newcommand{\Div}{\mbox{Div}}
\newcommand{\T}{\mathcal{T}}
\newcommand{\atan}{\operatorname{atan2}}
\newcommand{\acos}{\operatorname{acos}}


\begin{document}

\title{Advanced Calculus HW 11 - Due December 1, 4pm}
\author{Luis Berlioz}
\maketitle



\begin{prob}[\#9 page 198]
    Assume that $f: \R \to \R$ is differentiable.
    \begin{enumerate}[(a)]
        \item If there is an $L<1$ such that for each $x\in \R$ we have $f'(x)<L$, prove that there exists a unique point $x$ such that $f(x) =x$. [$x$ is a fixed point for $f$.]        
        \item Show by example that (a) fails if $L=1$.
    \end{enumerate}
\end{prob}
\begin{soln}
    \begin{enumerate}[(a)]
        \item If $f(0)=0$, then $x=0$ is a fixed point of $f$. In the case $f(0)>0$, then by the mean value theorem for all $x>0$ there exists a $0< c< x$ such that:
            $$f(x) - f(0) = f'(c)x < Lx < x$$
            This implies that:
            $$f(x) -x < f(0) +(L-1)x$$
            This shows that  in the interval $[0,f(0)/(1-L)]$, $f(x) -x$ is going from the positive value of $f(0)$ to some negative value at $x=f(0)/(1-L)$. By the intermediate value theorem, which $f(x) - x$ satisfies since it is continuous, $f$ must have a fixed point.

            Similarly for the case $f(0)<0$. For any $x<0$, by the mean value theorem:
            $$f(x) - f(0) = f'(c)x > Lx > x$$
            This implies:
            $$x-f(x) < (L-1)x -f(0)$$
            Therefore, $f$ has a fixed point in the interval $[f(0)/(1-L), 0]$.

            In every case the fixed point of $f$ is  unique because $(f(x)-x)' < L-1<0$. This means that the function is one to one so $f(x)-x$ can only be equal to zero once.

        \item Consider $f(x) = e^{-x } +x$. Then $f'(x) = -e^{-x }+1<1$ for all $x\in \R$. Observe that $f$ has no fixed point since the function: $f(x)-x = e^{-x }$ has no roots.
    \end{enumerate}
\end{soln}
\vspace{1in}



\begin{prob}[\#28 page 203   ]
    Suppose that $Z\subset \R$. Prove that the following are equivalent.
    \begin{enumerate}[(i)]
        \item $Z$ is a zero set.
        \item For each $\epsilon>0$ there is a countable covering of $Z$ by intervals $[a_i, b_i]$ with total length $\sum b_i - a_i <\epsilon$.
        \item For each $\epsilon>0$ there is a countable covering of $Z$ by set $S_i$ with total diameter $\sum diam\, S_i <\epsilon$.

    \end{enumerate}
\end{prob}
\begin{soln}
\begin{description}
    \item[(i) $\implies$ (ii)] The length of the open interval $]a,b[$ is the same as the length of the closed interval with the same limits $[a,b]$. This implies that for every $\epsilon >0$ $Z$ can be covered by open intervals $]a_i,b_i[$ for $i\in \N$ such that:
        $$\sum_{i\in \N }(b_i-a_i)<\epsilon $$
        The closed intervals $[a_i,b_i]$ for all $i\in \N$ also cover $Z$ and their lengths is also less than $\epsilon$.
    \item[(ii) $\implies$ (iii)] The diameter of a the closed intervals $S_i = [a_i, b_i]$ is $b_i -a_i$. If we take the same collection of intervals $[a_i,b_i]$ as above then we have a countable covering $S_i$ for all $i\in \N$ such that $\sum diam\, S_i <\epsilon$.
    \item[(iii)$\implies$ (i)] For any covering of $Z$ by sets $S_i$ for all $i\in \N$, for every $z\in Z$ there exists $S_j$ such that $z\in S_j$.  Let $a_j= \inf S_j$ and $b_j = \sup S_j$. This means that $diam \, S_j = b_j-a_j$ and that $S_j \subset ]a_j, b_j[$. Lastly, by hypothesis for all $\epsilon >0$  there exists a covering $S_i$  and sequences $a_i$ and $b_i$ defined as above such that:
        $$\sum (b_i - a_i) = \sum diam\, S_i < \epsilon$$
\end{description}
\end{soln}
\vspace{1in}


\begin{prob}[\# 37  page 205]
    Suppose that $f: \R \to \R$ has no jump dicontinuities. Does $f$ have the intermediate value property? (Proof of counterexample)
\end{prob}
\begin{soln}
    Consider the counterexample: $f(x) =sgn(x) + 1/2\sin(1/x)$ when $x\neq 0$ and $f(0)=0$. This function is continuous on all $\R$ except for $x=0$. And at  this point it has an oscillating discontinuity because we have shown in class that the limit when $x\to 0^+$ or $0^-$ of functions similar to $\sin(1/x)$ does not exist. Thus, we conclude that $f$ has no jump discontinuities. Also, $f$ does not have the intermediate value property since even though it has 0  and  1 in its range, it does not go through every number in [0,1], for example 1/4.
\end{soln}
\vspace{1in}


\begin{prob}[\# 46  page 206]
    \begin{enumerate}[(a)]
        \item Prove that the integral of the Zeno's staircase function described on page 174 is 2/3.
        \item What about the Devil's staircase?
    \end{enumerate}
\end{prob}
\begin{soln}
    \begin{enumerate}[(a)]
        \item Let $f(x)$ be the Zeno's staircase function, then it has a countable set of point where it is discontinuous, namely $\{1 - 1/2^n :\ n\in \N \}$. This means that the function is Riemann Integrable. To find the integral we sum up the areas under which the function is constant:
            $$\sum_{n=1 }^\infty \left(1-\frac 1{2^n}\right)\frac 1{2^n}= \sum_{n=1 }^\infty\frac 1{2^n} - \frac 1{4^n}$$
            And since both these series are convergent they can be summed individually to get $1- 1/3=2/3$.
        \item Using the terminology in the textbook, the devil's staircase is given by the function:
            $$H(x)=\sum_{i=1 }^\infty \frac{\omega_i/2}{2^i}$$
            When $x$ is in the Cantor set $C$. And $H$ has constant value at the discarded gap intervals. Also, $H$ is continuous therefore it is integrable. 

            To find the value of the integral, observe that for all $0\leq x \leq 1$, $H(x)+ H(1-x) =1$. If $x\in C$ then:
            $$H(x)+ H(1-x) = \sum_{i=1 }^\infty \frac{\omega_i/2}{2^i} + \sum_{i=1 }^\infty \frac{(2-\omega_i)/2}{2^i} = \sum_{i=1 }^\infty \frac{1}{2^i}= 1$$
            If $x$ is not in $C$ then it is in one of the discarded intervals which has endpoints that can be written in base 3 using only the symbols 0 and 2 so the above argument also applies. Finally, the integral is then:
            $$\int_0^1 H(x) dx + \int_0^1 H(1-x)dx = \int_0^1 1dx = 1$$
            Doing the change of variable: $u = 1-x$, $du = -dx$:
            $$\int_0^1 H(x) dx - \int_1^0 H(u)du = 2\int_0^1 H(x)dx = 1$$
And we get that the integral is 1/2.
    \end{enumerate}
\end{soln}
\vspace{1in}

\begin{prob}[\# 51  page 207]
    If $f,g$ are Riemann integrable on $[a,b]$,  and $f(x) < g(x)$ for all $x\in [a,b]$, prove that $\int_a^b f(x)dx < \int_a^bg(x)dx$. (Note the \textit{strict} inequality.)
\end{prob}
\begin{soln}
    Assume, to get a contradiction, that $\int_a^b(g(x)-f(x))dx = 0$. Then by corollary 31, $g(x) = f(x)$ almost everywhere. This is  a contradiction since the empty set has measure zero. 

    This means that $\int_a^b(g(x)-f(x))dx > 0$ equivalently:
    $$\int_a^bg(x)dx > \int_a^bf(x)dx = 0$$
\end{soln}
\vspace{1in}

\begin{prob}[\# 59  page 208]
    Prove that if $a_n\geq 0$ and $\sum a_n$ converges then $\sum(\sqrt{a_n})/n$ converges.
\end{prob}
\begin{soln}
   First note that for all $n\in \N$
    $$0\leq (\sqrt{a_n}-1/\sqrt n)^2 = a_n - 2\sqrt{a_n}/n + 1/n^2$$
    Then:
    $$\sqrt{a_n}/n\leq 1/2(a_n +1/n^2)$$
    Euler showed in 1734, that $\sum 1/n^2$ converges then $\sqrt{a_n}/n$ also converges.
\end{soln}
\vspace{1in}

\begin{prob}[\# 60  page 208]
    \begin{enumerate}[(a)]
        \item If $\sum a_n$ converges and $(b_n)$ is monotonic and bounded, prove that $\sum a_nb_n$ converges.
        \item If the monotonicity condition is dropped, or replaced by the assumption that $\lim_{n\to \infty }b_n =0$, find a counterexample to convergence of $\sum a_n b_n$.
    \end{enumerate}
\end{prob}
\begin{soln}
    \begin{enumerate}[(a)]
        \item Observe that $(b_n)$ converges because it is monotonic and bounded. Take $N\in \N$ such that if $n,m>N$ then:
            $$\left|\sum_{k=n }^m a_k \right| <\epsilon \quad \text{and} \quad |b-b_n|<\epsilon<|b|$$
            By the Cauchy criterion for sums. We can then write $a_nb_n = a_n(b-b+b_n)$ and thus
            $$\left|\sum_{k=n }^m a_kb_k \right| \leq |b|\left|\sum_{k=n }^m a_k \right| + \leftv|\sum_{k=n }^m a_k \right|$$

    \end{enumerate}

\end{soln}
\vspace{1in}



\end{document}








