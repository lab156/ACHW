\documentclass{article}
\usepackage[utf8]{inputenc}
\usepackage{amsmath,color,amssymb,amsthm,mathrsfs,verbatim,tikz,graphicx}
\usepackage[margin=2.5cm]{geometry}
\usetikzlibrary{matrix,arrows,decorations.pathmorphing}
\theoremstyle{definition}
\newtheorem{defn}{Definition}
\newtheorem*{fact}{Fact}
\newtheorem{example}{Example}
\newtheorem*{ex}{Exercise}
\newtheorem*{soln}{Solution}
\newtheorem*{prob}{Problem}
\newtheorem*{lemma}{Lemma}

\theoremstyle{theorem}
\newtheorem{thm}{Theorem}

\newcommand{\R}{\mathbb{R}}
\newcommand{\A}{\mathbb{A}}
\newcommand{\Q}{\mathbb{Q}}
\newcommand{\Z}{\mathbb{Z}}
\newcommand{\N}{\mathbb{N}}
\newcommand{\C}{\mathbb{C}}
\newcommand{\E}{\mathbb{E}}
\newcommand{\F}{\mathbb{F}}
\newcommand{\Proj}{\mathbb{P}}
\newcommand{\HP}{\mathbb{H}}
\newcommand{\D}{\mathbb{D}}
\newcommand{\Pic}{\mbox{Pic}}
\newcommand{\Div}{\mbox{Div}}
\newcommand{\T}{\mathcal{T}}

\begin{document}

\title{Advanced Calculus HW 1 - Due September 6}
\author{Luis Berlioz}
\maketitle



\begin{prob}[Lemma from Notes]
Prove the following lemma:
\begin{lemma}
If $X$ and $Y$ are in $\R^+$, then $X \leq Y$ if and only if $Y \subset X$.
\end{lemma}
\end{prob}
\begin{soln}
	First, we assume that $X,\ Y\in \R^+$ and $X\leq Y$. The inequality is equivalent to saying that for all $(y_1,y_2) \in Y$ there exists $(x_1,x_2) \in X$ such that $(x_1,x_2) < (y_1, y_2)$. Since we know that $X$ is right-complete and ``$(y_1, y_2)< (x_1,x_2)$ is false'', then $(y_1,y_2)\in X$. This means that $Y\leq X$.

	Next, assume that $Y\subset X$ (we are also assuming they are in $\R^+$). According to the left-openness property, for all $(y_1,y_2)$ there exists another element of $Y$ that we call $(y_1', y_2')$ and that satisfies:
	$$(y_1', y_2')<(y_1,y_2)$$
	We are working under that assumption that $Y$ is a subset of $X$, thus $(y_1', y_2')\in X$. Since the choice of $(y_1, y_2)$ was arbitrary, we get $X\leq Y$. 
\end{soln}
\vspace{1in}


\begin{prob}[Reals have LBP]
Prove that the reals, as described, have the lower bound property: any set of positive reals has a greatest lower bound.
\end{prob}
\begin{soln}
	First, we will show that the arbitrary union of  elements in $\R^+$ is also in $\R^+$. Let $I$ be any index set and $ X_i\in \R^+$ for all $i\in I$:
	$$ X = \cup_{i\in I} X_i $$
	We must show two things:
	\begin{description}
		\item[$\cup_{i\in I} X_i$ is left-open: ] Let $(p,q) \in \cup_{i\in I} X_i $, then it must be the case that $(p,q)\in X_i$ for some $i \in I$. Then, using that $X_i$ is left-open, there exists $(r,s) \in X_i \subset \cup_{i\in I} X_i$ such that $(r,s)<(p,q)$.  
		\item[$\cup_{i\in I} X_i$ is right-complete:] Suppose that for some $(p,q) \in \cup_{i\in I} X_i$ we have that $(s,t)\geq (p,q)$. Since $(p,q)\in X_i$ for some $i\in I$ and we know that $X_i$ is right complete, then we get that $(s,t)\in \cup_{i\in I} X_i$ and thus it is also right-complete.
	\end{description}
	Let $Y$ be a lower bound for the set $\{ X_i :\ i\in I\}$ such that:
	$\cup_{i\in I} X_i \leq Y$ this is equivalent to $\cup_{i\in I} X_i \supset Y$.

	On the other hand, since $Y\leq X_i$ for all $i\in I$, we have that $Y\supset X_i$ for all $i\in I$. But this means that $Y\supset \cup_{i\in I} X_i$ which is equivalent to $Y\leq \cup_{i\in I} X_i$. Therefore, $Y= \cup_{i\in I} X_i$, in other words, $\cup_{i\in I} X_i$ is the largest lower bound.
\end{soln}
\vspace{1in}



\begin{prob}[Empty set and $\N^2$]
Analyze and discuss properties of the emptyset and $\N^2$.
\end{prob}
\begin{soln}
	\begin{itemize}
		\item The empty set $\emptyset$ corresponds to infinity, this is because it vacuously satisfies that $X< \emptyset$ for all positive real numbers $X$. And is a right-complete and left-open set  for the same reason.
		\item On the other hand, $\N^2$ corresponds to zero. In order to prove this assertion, we have to prove that $\N^2$ satisfies the properties of the identity element for the monoid: $(\R^+\cup\{0\},+)$. $\N^2$ is right complete since any pair greater than another is of course an ordered pair in $\N^2$. Similarly, for any pair $(p,q)$ there exists $(1,q+1)$ which is smaller since: $(q+1)p > q$, which means $\N^2$ is left-open. Finally, for any real positive number $X$:
			$$\N^2 + X = \{z > x+y;\  y\in \N^2 , x\in X\} = X$$
			To show this, take any $(p,q)\in X$ we will find $(x_1,x_2)\in X$ and $(1,n)\in \N^2$ such that:
			$$(p,q) > (x_1,x_2)+ (1,n)$$
			Since $X$ is left-open, take $(x_1, x_2)$ to be such that $(p,q)>(x_1,x2)$, equivalently: $px_2> qx_1$, the inequality is strict, then there is $m\in \N$ such that:
			$$px_2 = qx_1 + m$$
			Equivalently, multiplying $n$ on both sides:
			$$px_2n = qx_1n + mn$$
Now take $n\in \N$ such that $nm>qx_2$ this implies that:
$$ px_2n > qx_1n + qx_2$$
			Writing this as ordered pairs means $(p,q) > (x_1,x_2)+(1,n)$. Therefore $X\subset X+\N^2$. The other inclusion is much simpler since right completeness implies that for all $(x_1,x_2)\in X$ and for all $(p,q)\in N^2$ we have:
			$$(x_1,x_2)<(x_1,x_2) + (p,q)$$
			Then $(x_1,x_2) + (p,q) \in X$ implying that $X+\N^2 \subset X$.
	\end{itemize}
\end{soln}
\vspace{1in}


\begin{prob}[Negative Reals]
Dream up a nice way to incorporate the negative real numbers into this setting. [Hint: Use an algebraic approach.]
\end{prob}
\begin{soln}
	To get the reals from the positive reals, first add $\N^2$ which according to the above acts as an additive identity. This gives us the nonnegative reals, let's call them $\R_0$. Take the set $\R_0\times \R_0$ of ordered pairs of nonnegative reals and for $(x,y), (z,w) \in \R_0\times \R_0$ define the operations:
	\begin{itemize}
		\item $(x,y) + (z,w) = (x+z,y+w)$
		\item $(x,y)*(z,w) = (xz+yw, xw+yz)$
	\end{itemize}
	And the equivalence relationship $(x,y) \sim (x', y') \iff (x',y')=(x,y)+(z,z)$ for some  $z\in \R_0$.
	
	Define the reals to be $\R = (\R_0\times \R_0)/\sim $. We denote the equivalence class of $(\N^2, \N^2)$ as 0 for brevity and clarity. 
	\begin{itemize}
		\item Addition is well-defined since, if we take two pairs from the equivalence class of $(x+x', y+y')$ that is:
			$$(a,b), (a',b') \in [(x+x', y+y')]$$
			Then $(a,b) = (x+x', y+y')+(z,z) \text{ and } (a',b') = (x+x', y+y')+(z',z') $ for some $z,z'\in \R_0$. This implies that:
			$$(a,b) \sim  (x+x', y+y')+(z,z)+(z',z')$$
			$$(a',b') \sim  (x+x', y+y')+(z',z')+(z,z)$$
		Therefore:	
			$$(a,b) \sim  (x+x', y+y')+(z+z',z+z') \sim (a', b')$$
		\item Similarly for the product:
$$(x+s,y+s)*(z+s',w+s') = (xz+yw +xs'+sz+ss'+ys'+sw+ss', xw+yz+ys'+sz+ss'+xs'+sw+ss')$$
Cancelling similar terms gives:
			$$(x+s,y+s)*(z+s',w+s') \sim (xz+yw, xw+yz)$$
	\end{itemize}
	Assuming that the positive reals with multiplication are an abelian group $(\R^+)$ and that addition and multiplication for the positive reals satisfy the distributive property. (I will not prove this since it would take a looong time), all that we need to show in order to prove $(\R, +,\cdot)$ is a field is that $(\R,+)$ is an abelian group: we have already established the nonnegative reals a commutative monoid, we just need to check for the additive inverse. Thus: $$(x,y)+(y,x)= (x+y, y+x) \in 0$$
	 It is also an ordered field with the additional property that:
	$$(x,\N^2)<(y,\N^2) \implies (\N^2,y)< (\N^2,x)$$
	Finally, a well known theorem says that any ordered field with the greatest lower bound property is isomorphic to the real numbers.
\end{soln}
\vspace{1in}


\begin{prob}[Prelim Problem]
Consider $f(x) = \frac{\ln x}{x}$ which is defined for all positive real numbers. Find, with proof, the Taylor series of $f$ centered at $x=1$. Also find, with proof, the set of all real $x$ for which the Taylor series converges, and (if you are familiar with uniform convergence) the subset of reals for which the series converges uniformly.
\end{prob}
\begin{soln}
	In order to find the Taylor series for $ f(x) = \ln x/x$ first observe that:
	$$\frac 12\frac d{dx}(\ln x)^2 = f(x)$$ 
	And the Taylor series of $\ln x$ is much easier to compute since:
	$$\frac{f^{(n)}}{n!} = \frac{(-1)^{n+1}(n-1)!}{n!} = \frac{(-1)^{n+1}}{n}$$
	The series for $\ln x$ around $x=1$ is then:
	$$\ln(x) = \sum_{n=1}^\infty \frac{(-1)^{n+1}}{n}(x-1)^n $$
	Squaring and Differentiating both side of the above equation: 
	$$\frac {\ln x}{x} = \left[\sum_{n=1}^\infty \frac{(-1)^{n+1}}{n}(x-1)^n\right]\left[ \sum_{n=1}^\infty (-1)^{n+1} (x-1)^{n-1}\right]$$
	Using the formula for the product of Taylor series, the  coefficients of the $(x-1)^n$ term is:
	$$\sum_{k=1}^n \frac{(-1)^{k+1}}{k}(-1)^{n-k} =\sum_{k=1}^n \frac{(-1)^{n+1}}{k}$$ 

	$$\frac{\ln x}x = \sum_{n=1}^\infty (-1)^{n+1}\left[ \sum_{k=1}^n \frac 1k\right] (x-1)^n$$
	According to the ratio test the series converges in the interval:
	$$\lim_{n\to \infty} \frac{1+\ldots + \frac 1{n+1}}{1+\ldots + \frac 1n}|x-1| = \lim_{n\to \infty} \left( 1+ \frac n{n+1} \right) |x-1| = |x-1| <1 $$
	That is: $0<x<2$. This interval is open on both sides since $f$ is unbounded on zero.  And on $x=2$ has the sum of alternating harmonic series terms as coefficient on $x=2$. These numbers for a divergent sequence: If we call $H_k=\sum_{j=1}^k 1/j$ then $\sum_{k=1}^n (-1)^k H_k$ diverges.

	As for all power series with open interval of convergence (in this case that radius of convergence is $R=1$), we have uniform convergence for all intervals of the form: $]\epsilon, 2-\epsilon[$ for all small $1\geq \epsilon>0$

\end{soln}
\vspace{1in}


\begin{prob}[Sequence with limit $\sqrt{2}$]
Consider the sequence of rational numbers defined recursively by the following formula: $$(x,y) \mapsto (x^2 + 2y^2, 2xy)$$ and starting at the point $(x,y) = (2,1)$. Compute the first 8 terms of this sequence as fractions, and evaluate their squares as decimals. Show that the limit of the resulting sequence is $\sqrt{2}$.
\end{prob}
\begin{soln}
The first 8 terms are:
	\begin{center}
	\begin{tabular}{|r|r|}
		\hline
		$n$ & $a_n$ \\
		\hline
1& 2.00000000000000\\
		\hline
2& 1.50000000000000\\
		\hline
3& 1.41666666666667\\
		\hline
4& 1.41421568627451\\
		\hline
5& 1.41421356237469\\
		\hline
6& 1.41421356237310\\
		\hline
7& 1.41421356237310\\
		\hline
8& 1.41421356237310\\
\hline
	\end{tabular}
	\end{center}

	To show that the sequence of fractions obtained recursively converges to $\sqrt 2$, first we write it as:
	$$\frac{x^2 + 2y^2}{2xy} = \frac x{2y} + \frac yx $$
	If we let $a_n = x/y$ for each term in the sequence, we get:
	$$a_{n+1} = \frac{a_n}{2} + \frac 1{a_n}$$
	Observe that $a_n > \sqrt 2$ for all $n>0$ since starting with $a_1=2$ and assuming that $a_n^2>2$ implies:
	$$a_{n+1}^2-2 = \left( \frac{a_n}2 + \frac 1{a_n}\right)^2 - 2 = \frac{a_n^2}4 - 1 + \frac 1{a_n^2} = \frac{a_n^4 - 4a_n^2 +4}{4a_n^2} = \frac{(a_n^2-2)^2}{4a_n^2} > 0$$
	Also note that $a_n $ defines a decreasing sequence, that is $a_{n+1} < a_n$ for all $n\geq 1$: 
	$$a_n - a_{n+1} = a_n-\left(\frac{a_n}{2} + \frac 1{a_n}\right)=\frac{a_n}{2} - \frac 1{a_n} = \frac{a_n^2 - 2}{2a_n}>0 $$
	This means that the sequence $\{a_n\}$ is decreasing and bounded. Thus, converges to its g.l.b.: $\inf_{n\in \N\}}\{a_n\} = a$. If this was not the case, there would be an $\epsilon>0$ such that $|a_n - a| \geq \epsilon$ for all $n\in \N$ contradicting the assumption that $a$ is the greatest lower bound.
	
	To finish the proof, note that the recursion step is given by the continuous function $f(x) = x/2 + 1/x$ on $x>0$; and that this function has only fixed points. That is, the equation $x = f(x)$ has only two solutions: $\pm \sqrt 2$. Finally, we show that $a$ ( the limit of $a_n$) has to be a fixed point of $f(x)$:
	$$0 = \lim_{n\to \infty}(a_{n+1} - a_n) = \lim_{n\to \infty}(f(a_n) - a_n) =f(a) - a $$
	Since there are only two fixed points of $f(x)$ and $a$ is positive, $a$ has to be $\sqrt 2$.

\end{soln}
\vspace{1in}


\end{document}
